% Metódy inžinierskej práce

\documentclass[10pt,twoside,slovak,a4paper]{article}
\usepackage {float}
\usepackage[slovak]{babel}
%\usepackage[T1]{fontenc}
\usepackage[IL2]{fontenc} % lepšia sadzba písmena Ľ než v T1
\usepackage[utf8]{inputenc}
\usepackage{graphicx}
\usepackage{url} % príkaz \url na formátovanie URL
\usepackage{hyperref} % odkazy v texte budú aktívne (pri niektorých triedach dokumentov spôsobuje posun textu)

\usepackage{cite}
%\usepackage{times}

\pagestyle{headings}

\title{Vývin inovatívnej platformy e-learningu pre nepočujúcich\thanks{Semestrálny projekt v predmete Metódy inžinierskej práce, ak. rok 2020/21, vedenie: Ing. Jozef Sitarčík}} % meno a priezvisko vyučujúceho na cvičeniach

\author{Petra Hlavinová\\[2pt]
	{\small Slovenská technická univerzita v Bratislave}\\
	{\small Fakulta informatiky a informačných technológií}\\
	{\small \texttt{xhlavinova@stuba.sk}}
	}

\date{\small 15. október 2020} % upravte



\begin{document}

\maketitle

\begin{abstract}
E-learning je čoraz viac atraktívnejším nástrojom pre študentov. Vďaka neustálemu rozširovaniu inteligentných zariadení a modernej technológie sa stáva učenie a získavanie informácií stále jednoduchšie a rýchlejšie. Práve pre nepočujúcich študentov môže mať e-learning množstvo výhod, no napriek tomu čelí táto skupina so špecifickými vzdelávacími potrebami v dnešnej dobe mnohým prekážkam v prístupe ku vzdelávaniu. V tejto štúdii sme sa zamerali na skúmanie týchto prekážok, spôsobu, akým sa nepočujúci učia, ako aj na to, ako by im mali byť prezentované informácie a najmä vzdelávací obsah. Naším cieľom bolo analyzovať kognitívne vlastnosti nepočujúcich a hlavne nájsť a vyvinúť inovatívnu a užívateľský prívetivú platformu e-learningu vhodnú pre potreby nepočujúcich študentov. 
\end{abstract}

\section{Úvod}

\begin{center}
Zdá sa, že nepočujúci ľudia sú v dnešnej dobe stále do veľkej miery sociálne vylúčení. Sociálne vylúčenie nepočujúcich je podmienené viacerými faktormi, či už vzdelávacím systémom, právnymi predpismi v sociálnej oblasti a postojmi ľudí v spoločnosti. Paradoxne však práve celoživotné vzdelávanie predstavuje rozhodujúci parameter proti sociálnemu vylúčeniu nepočujúcich ľudí. Vznik e-learningu uľahčil vzdelávanie sa ľuďom na celom svete. Vplyvu internetu na skupinu sluchovo postihnutých sa budeme venovať v časti~\ref{vplyv}. Sila internetu, a teda aj e-learning, spočíva v jeho univerzálnosti. Z toho vyplýva, že musí byť prístupný ľuďom s rôznym rozsahom sluchu, zraku, pohybu a kognitívnych schopností. V minulosti sa však zvyklo myslieť, že nepočujúcim stačí na uchopenie informácií písomný prepis zvukového obsahu. Práve toto predstavuje hlavný problém pri vzdelávaní tejto cieľovej skupiny, ktorému sa budeme potrebnejšie venovať v kapitole číslo~\ref{rozdiely}. Práve preto, aby e-vzdelávanie pokrylo túto medzeru v gramotnosti, sú potrebné kreatívne spôsoby prezentácie vzdelávacích materiálov. Vytváraniu vhodnej platformy e-learningu sa budeme venovať v časti~\ref{navrhovanie}.
\end{center}


\section{Vplyv internetu na skupinu sluchovo postihnutých} \label{vplyv}

\begin{flushleft}
Podľa World Wide Web Consortium\footnote{World Wide Web Consortium (W3C) je medzinárodné konzorcium, ktorého členovia spoločne s verejnosťou vyvíjajú webové štandardy pre World Wide Web.\cite{WWWC}} bol internet navrhnutý tak, aby fungoval pre všetkých jednotlivcov bez ohľadu na ich schopnosti. Vplyv internetu zásadne mení prístup k vzdelávaniu pre nepočujúcich, pretože jeho použitie \emph{odstraňuje komunikačné a interakčné bariéry}, ktorým môžu jednotlivci vo fyzickom svete čeliť. To zdôrazňuje potrebu poskytnúť ľuďom so zdravotným postihnutím príležitosti bez ohľadu na ich zdravotné postihnutie sprístupnením online služieb, najmä e-learningu, spôsobom, ktorý rešpektuje a zohľadňuje ich potreby. Nejde len o rovnaké ľudské práva, ale aj o to, aby táto technológia a jej výhody mali mať úžitok aj pre nepočujúcich a sluchovo postihnutých ľudí. Inak provokuje fenomén „digitálnej priepasti“.\cite{pappas2018learning}
\end{flushleft}



\section{Rozdiely v kognitívnych funkciách medzi nepočujúcimi a počujúcimi jedincami} \label{rozdiely}
Vývin vhodnej e-learningovej platformy pre nepočujúcich by sa mal realizovať v súlade s učebným profilom, ako aj so špecifickými vzdelávacími potrebami  tejto cieľovej skupiny. Z tohto dôvodu je potrebné v prvom rade preskúmať hlavné rozdiely v kognitívnych schopnostiach medzi nepočujúcimi a počujúcimi jedincami.\linebreak
Tieto rozdiely môžeme rozdeliť do následovných sfér:
\begin{itemize}
\item ~\ref{rozdiely:vseob} Všeobecný kognitívny profil
\item ~\ref{rozdiely:pozornost} Pozornosť
\item ~\ref{rozdiely:pamat} Práca s pamäťou
\item ~\ref{rozdiely:citanie} Čítanie s porozumením
\end{itemize}



\subsection{Všeobecný kognitívny profil} \label{rozdiely:vseob}
Priemerné skóre IQ nepočujúcich je porovnateľné s počujúcimi jednotlivcami a má tendenciu sa časom zvyšovať. Nepočujúci jedinci však vykazujú rozdiely v porovnaní s počujúcimi rovesníkmi, pokiaľ ide o ich pamäť, zručnosti pri riešení problémov a akademické výsledky. Vizuálne komunikačné schopnosti nepočujúcich, ako napríklad spracovanie vizuálneho jazyka, predstavujú individuálne kognitívne rozdiely. Uvádza sa, že nepočujúci a sluchovo postihnutí jedinci majú rovnakú úroveň citlivosti na vizuálny kontrast. Okrem toho sa zdá, že nepočujúce deti majú poruchy jemného motorického sekvenovania, a to napriek skutočnosti, že ich vizuopriestorové kognitívne schopnosti sa výrazne nelíšia od ich sluchových vrstovníkov.
\cite{pappas2018learning}

\subsection{Pozornosť} \label{rozdiely:pozornost}
Pozornosť je rozhodujúca pre každodenné činnosti nepočujúcich jedincov, pretože identifikácia periférnych vizuálnych znakov je pre nich dôležitejšia z dôvodu straty sluchu. Strata sluchu sa nepovažuje za prediktívny faktor deficitov pozornosti, pretože nepočujúci jedinci majú v porovnaní so svojimi počujucími rovesníkmi rovnaké a niekedy dokonca aj lepšie výsledky pri riešení úloh vyžadujúcich pozornosť. Avšak existujú aj dôkazy o významných rozdieloch vo vizuálnej pozornosti, najmä v periférnej pozornosti\cite{pappas2018learning}. Štúdia Baveliera, Dyeho a Hausera (2006) odhalila, že nepočujúci jedinci sú viac rozptýlení periférnymi rušičmi a menej centrálnymi rušičmi. \cite{bavelier2006deaf} 

\subsection{Práca s pamäťou} \label{rozdiely:pamat}
Zdá sa, že včasné vystavenie posunkovej reči má pozitívny vplyv na kognitívny vývoj nepočujúcich detí, ako aj na zvládnutie vizuálnych perspektív. Napriek tomu však existujú dôkazy, že medzi nepočujúcimi a počujúcimi jedincami sú rozpoznateľné rozdiely pri práci s pamäťou. Tieto rozdiely vyplývajú zo skutočnosti, že zapamätávanie pomocou posunkového jazyka vyžaduje viac miesta v pamäti ako prostredníctvom hovoreného jazyka. Pre predstavu ako vyzerá abeceda v posunkovej reči nahliadnite na obrázok ~\ref{slovenskaposunkovarec} v časti ~\ref{navrhovanie} . Posunkové jazyky môžu mať negatívny vplyv na krátkodobú pamäť kvôli ich vizuopriestorovej povahe. 
\cite{pappas2018learning}.

\subsection{Čítanie s porozumením} \label{rozdiely:citanie}
Nepočujúci majú nízku úroveň čitateľských schopností, pretože majú odlišný spôsob rozpoznávania slov v porovnaní s počujúcimi čitateľmi. Porozumenie textu nepočujúcich dospelých súvisí s ich čitateľskou motiváciou, a tak by práve náročné materiály na čítanie mohli postupne zlepšovať ich čitateľské schopnosti. Domínguez a Alegria (2009) skúmali mechanizmy čítania, ktoré používajú nepočujúci dospelí. Výsledky ukázali, že väčšina účastníkov používa na pochopenie textu stratégiu kľúčových slov. \cite{pappas2018learning} \cite{dominguez2010reading}

\section{Navrhovanie e-learningových systémov} \label{navrhovanie}
\subsection{Problémy pri vytváraní e-learningových rozhraní} \label{navrhovanie:problemy}
Špeciálne vzdelávacie potreby osôb so zdravotným postihnutím, ako je hluchota a čiastočná strata sluchu, sa žiaľ dodnes pri vývoji systémov elektronického vzdelávania zohľadňujú zriedka. Je pravda, že návrh rozhraní, ktoré sú pre nich vhodné a užívateľsky prívetivé, nie je vždy ľahký proces. Jazykové a gramotné schopnosti sa môžu líšiť v závislosti od typu a úrovne hluchoty a veku človeka, ktorý stratil sluch, a môžu byť ovplyvnené zároveň aj ich schopnosti čítania a písania. To vyvoláva problémy pri vytváraní e-learningových rozhraní. Projektanti takýchto systémov musia zohľadňovať špeciálne potreby cielovej skupiny, ktoré sa vyskytujú na komunikačnej aj kognitívnej úrovni.\paragraph{Posunkový jazyk}- Nepočujúci majú k dispozícii špeciálny dorozumievací jazyk - posunkový jazyk, ktorý namiesto zvukov využíva manuálne prostriedky komunikácie, reč tela, pohyby hlavy a hornej časti trupu.
\begin{figure}[H]
   \includegraphics[scale=0.1]{dvojrucna-tlac.jpg}
\centering
\caption{Slovenská prstová abeceda - dvojručná.}
\cite{mytyafakty}
\label{slovenskaposunkovarec}
\end{figure}
\subsection{Ako na to?}\label{navrhovanie:ako}
Pri navrhovaní e-learningových systémov pre nepočujúcich a sluchovo postihnutých jedincov je potrebné zabezpečiť všetok zvuk vizuálnym spôsobom pomocou textu, titulkov, obrázkov a videí v posunkovej reči a tiež vytvoriť grafické rozhranie, ktoré efektívne a zrozumiteľne prezentuje výučbu Používanie textu by sa malo obmedziť na minimum, pretože nepočujúci ľudia majú do istej miery ťažkosti s čítaním s porozumením.Štúdie napríklad ukazujú, že nepočujúci, ktorí používajú posunkovú reč, spracúvajú obrázky ľahšie a efektívnejšie v porovnaní so slovami. Dizajnér a vývojár systému e-vzdelávania pre nepočujúcich študentov musí zohľadniť všetky vyššie uvedené parametre. \par Podľa najmodernejších poznatkov najdôležitejšie inštruktážne metódy v e-vzdelávaní zahŕňajú použitie príkladov, praktických otázok a spätnej väzby. \cite{pappas2018learning}

\section{Výsledky štúdii a dotazníku} \label{dolezitejsia}
\ldots
\section{Záver} \label{zaver} % prípadne iný variant názvu
\ldots
%\acknowledgement{Ak niekomu chcete poďakovať\ldots}

% týmto sa generuje zoznam literatúry z obsahu súboru literatura.bib podľa toho, na čo sa v článku odkazujete
\bibliography{literatura}
\bibliographystyle{plain} % prípadne alpha, abbrv alebo hociktorý iný
\end{document}
